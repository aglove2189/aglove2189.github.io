%%%%%%%%%%%%%%%%%%%%%%%%%%%%%%%%%%%%%%%%%
% Awesome Resume/CV
% XeLaTeX Template
% Version 1.2 (27/3/2017)
%
% This template has been downloaded from:
% http://www.LaTeXTemplates.com
%
% Original author:
% Claud D. Park (posquit0.bj@gmail.com) with modifications by
% Vel (vel@latextemplates.com)
%
% License:
% CC BY-NC-SA 3.0 (http://creativecommons.org/licenses/by-nc-sa/3.0/)
%
% Important note:
% This template must be compiled with XeLaTeX, the below lines will ensure this
%!TEX TS-program = xelatex
%!TEX encoding = UTF-8 Unicode
%
%%%%%%%%%%%%%%%%%%%%%%%%%%%%%%%%%%%%%%%%%

%----------------------------------------------------------------------------------------
%	PACKAGES AND OTHER DOCUMENT CONFIGURATIONS
%----------------------------------------------------------------------------------------

\documentclass[12pt, letterpaper]{awesome-cv} % A4 paper size by default, use 'letterpaper' for US letter

\geometry{left=2cm, top=1.5cm, right=2cm, bottom=2cm, footskip=.5cm} % Configure page margins with geometry

\fontdir[fonts/] % Specify the location of the included fonts

% Color for highlights
\colorlet{awesome}{awesome-darknight} % Default colors include: awesome-emerald, awesome-skyblue, awesome-red, awesome-pink, awesome-orange, awesome-nephritis, awesome-concrete, awesome-darknight
%\definecolor{awesome}{HTML}{CA63A8} % Uncomment if you would like to specify your own color

% Colors for text - uncomment and modify
%\definecolor{darktext}{HTML}{414141}
%\definecolor{text}{HTML}{414141}
%\definecolor{graytext}{HTML}{414141}
%\definecolor{lighttext}{HTML}{414141}

\renewcommand{\acvHeaderSocialSep}{\quad\textbar\quad} % If you would like to change the social information separator from a pipe (|) to something else

\name{Aaron}{Glover}
\mobile{281.221.5071}

\email{aglove2189@gmail.com}
\homepage{aglove2189.github.io}
\github{aglove2189}
\linkedin{aglove2189}
\twitter{@aglove2189}

\position{Data Scientist, Machine Learning Engineer}
\quote{I am value driven, motivated to maximize productivity through simplicity. I take a liberal arts approach to data science, refining decision-making across any organizational level. I find the answers in the questions, clarifying the “why” and predicting “what’s next”.}
\makecvfooter{}{\today}{\thepage} % Specify the letter footer with 3 arguments: (<left>, <center>, <right>), leave any of these blank if they are not needed

\begin{document}

\makecvheader

\cvsection{Skills}

\begin{cvskills}

  \cvskill
    {Languages \& Packages} % Category
    {Python, SQL, pandas, numpy, scikit-learn, keras, pytorch, matplotlib, streamlit, shap, mlflow, Airflow} % Skills

  \cvskill
    {Platforms} % Category
    {Docker, Kubernetes, Airflow, AWS, GCP, git, Azure Pipelines, GitHub Actions} % Skills

  \cvskill
    {Areas of Expertise}
    {DevOps, Code Optimization, Code Productionization, Python Packaging, Modeling}

  \cvskill
    {Soft Skills}
    {Leadership, Project Management, Strategic Planning, Problem Solving, Communication}

\end{cvskills}
\cvsection{Experience}

\begin{cventries}

\cventry
{Data Science Manager - Commercial} % Job title
{Enterprise Products} % Organization
{Houston, TX} % Location
{Feb 2022 - Current} % Date(s)
{ % Description(s) of tasks/responsibilities
\begin{cvitems}
\item {Focused on developing predictive models to assist with commodity trading and decision making. Since inception, over 50 models have been productionized across NGLs, Crude, Natural Gas, spreads, and cross commodities. Assets under management are in the multi-millions.}
\item {As the technical lead, I have lead the team to build a number of production grade systems and frameworks to assist modeling efforts, including a feature selection process, a 'no-code' model deployment system, and a model evaluation framework.}
\end{cvitems}
}

\cventry
{Lead Data Scientist} % Job title
{} % Organization
{} % Location
{May 2019 - Feb 2022} % Date(s)
{ % Description(s) of tasks/responsibilities
\begin{cvitems}
\item {Tasked with developing predictive models to assist with commodity trading. The inception of the project started with a team of two and has now grown to eight.}
\item {Developed a recurrent neural net (RNN) for predicting heat exchanger failures. Model was the first deployed solution at Enterprise, implemented in the first 8 weeks of hire.}
\item {Built a Monte Carlo simulation that optimized spare parts inventory, savings in the multi-millions.}
\end{cvitems}
}


\cventry
{Machine Learning Engineer} % Job title
{Sanchez Energy} % Organization
{Houston, TX} % Location
{Apr 2017 - May 2019} % Date(s)
{ % Description(s) of tasks/responsibilities
\begin{cvitems}
\item {Developed a model fitting solution for determining a well's spontaneous (SP) log curve using peak detection methods and Kalman filters. The end result was used for identifying potential oil field plays to target.}
\item {Developed a Markov Chain Monte Carlo (MCMC) solution for simulating a well's decline curve and ultimate recovery. This augmented Engineering's decision making on how much a well will produce over its lifetime.}
\item {Optimized an in house developed geophysics simulator in Python which decreased runtime by 6x and lines of code were reduced 10x.}
\item {Contributed to the development of a multi model prediction framework for predicting well production. The solution was a significant improvement on the industry standard decline curve fitting.}
\item {Implemented a real time alert for detecting tubing leaks which resulted in a cost savings in the six figures. An industry standard deterministic model was required by Engineering, the model was optimized by sampling the search space with a Tree-structured Parzen Estimator.}
\end{cvitems}
}


\cventry
{Data Analytics / BI Engineer} % Job title
{Occidental Petroleum} % Organization
{Houston, TX} % Location
{Jan 2012 - Mar 2017} % Date(s)
{
\begin{cvitems}
\item {Developed a Monte Carlo simulation to determine the optimal number of workover rigs for a given field. Implemented in fields across Texas and California with a savings in the high six figures.}
\item {Developed over 350 SSRS reports and Spotfire dashboards over the course of two years.}
\item {Designed and developed multiple SSAS cubes for operational and well servicing data, query times were reduced 1,000x.}
\item {Maintained and enhanced the main Operational Data Store (ODS) used company wide for production reporting.}
\item {Automated the delivery of partner reports utilizing SSRS and SQL which resulted in an 80\% reduction in man hours.}
\end{cvitems}
}

\end{cventries}
\cvsection{Education}

\begin{cventries}

\cventry
{Major GPA: 3.8} % Degree
{Texas A\&M | M.S. in Analytics} % Institution
{Houston, TX} % Location
{May 2017} % Date(s)
{ % Description(s) bullet points
\begin{cvitems}
\item {Thesis: Predicting the likelihood of ESP well failures utilizing survival analysis and gradient boosting.}
\end{cvitems}
}

\cventry
{Major GPA: 4.0} % Degree
{University of North Texas | B.S. in Information Systems} % Institution
{Denton, TX} % Location
{Dec 2011} % Date(s)
{}
\vspace{-\baselineskip}

\end{cventries}
\cvsection{Projects}
\begin{cventries}
\vspace{-\baselineskip}

\cventry{}{}{}{}
{
    \begin{cvitems}
    \item \href{https://github.com/aglove2189/appias}{\faExternalLink\acvHeaderIconSep appias}{ - A library for gluing together a few of the standard steps when exploring a dataset and building a model.}
    \item \href{https://twitter.com/taleb_gpt2}{\faTwitter\acvHeaderIconSep Taleb but AI}{ - GPT2 model trained on Nassim Taleb's quotes.}
    \item \href{https://github.com/aglove2189/cookiecutter_ds}{\faExternalLink\acvHeaderIconSep cookiecutter\_ds}{ - Repository template for starting data science projects.}
    \item \href{https://github.com/aglove2189/awair}{\faExternalLink\acvHeaderIconSep Awair}{ - Python library for viewing / downloading Awair data.}
    \end{cvitems}
}

\end{cventries}

\end{document}